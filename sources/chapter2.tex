%!TEX root = ../um-thesis.tex

\chapter{Overview}

\section{Compilation}

The recommended method of typesetting is to use \texttt{latexmk} which handles acronyms via the \texttt{latexmkrc} file.

\section{Fonts}

Fonts are configured in \texttt{preamble.tex} via the Komascript font settings, using the fontspec and unicode-math packages.

\begin{equation}
	e^\varphi + 5 = 0
\end{equation}

\section{Citations}

Citations are handled with \href{https://www.ctan.org/pkg/biblatex}{\texttt{biblatex}} and \texttt{biber} as backend. Different citation styles can be used by changing the arguments when loading the biblatex package. Example citation: \cite{Shannon:1949ti}.

\section{Acronyms}

Acronyms are handled via the \href{https://www.ctan.org/pkg/glossaries}{\texttt{glossaries}} package that automatically adds acronyms to a list of acronyms. Example: \gls{um} is located in \gls{aa}. \gls{um} is located in \gls{aa}. 

\section{Figures}

Tikz figures are pre-compiled via the \texttt{external} command. To name a figure, use the \texttt{tikzsetnextfilename} command.

\begin{figure}[ht]
	\tikzsetnextfilename{example_picture}
	\begin{center}
	\begin{tikzpicture}[
    	mybox/.style = {draw, rounded corners, minimum height=1.2cm, text width=2cm, align=center}
  		]
		\node[mybox, fill=green!70!black] (i1) {Item 1};
		\node[mybox, fill=blue!50, right=of i1] (i2) {Item 2};
		\draw[-latex, thick] (i1) -- node[above] {$x$} (i2);
	\end{tikzpicture}
	\end{center}
	\caption{Example illustration.}
\end{figure}

\section{List of Todos}

The \href{https://www.ctan.org/pkg/todonotes}{todonotes} package is a convenient way to keep a \todo{Improve wording here}{list of todos during} the writing.  

\todo[inline, color=blue!40]{Re-read this part}

Todos can be hidden by loading the package the \texttt{disable} argument.